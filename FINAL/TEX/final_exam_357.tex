\documentclass[10pt]{article}
\usepackage{amssymb,amsmath}
\usepackage{graphicx,mathrsfs}
\pagestyle{empty}

\usepackage[left=2cm, top=2cm, bottom = 2cm, right=2cm, nohead, nofoot]{geometry}

\def\a{\bf{a}}
\def\b{\bf{b}} 
\def\nn{\nonumber} 
\def\R{\mathbb{R}}

\begin{document}

\begin{center}
{\Large\bf Final Exam - MTH 357}\\
\vspace{.1in}
Due: 12/16/21 at 10:30 am (by email)\\

\end{center}

\noindent Name:\\

\noindent \textbf{Instructions:} This is a take-home exam.  You may use any books, notes, or other reference material.  You may \underline{NOT} seek help from \underline{ANY} other person or online solution. Of course, you are encouraged to ask me for hints.  Violations will be dealt with harshly. Please write solutions in a neat, organized, and easily readable manner.  Of course, show all your work and explain your answers as necessary.  If you use Mathematica or something similar for your integrals and computations, include appropriate documentation. Early submissions are encouraged as well. \\


\begin{enumerate}

\item (35 pts) The velocity field of a fluid with density $\rho = 400\ \tfrac{kg}{m^3}$ is given by $\textbf{F}(x,y,z) = \left<1-x,-y,-z-1\right>$ $\tfrac{m}{s}$. Let $T$ be the polyhedral region with vertices $(0,0,0)$, $(1,0,0)$, $(0,1,0)$, and $(0,0,1)$ (see below).  $S$ is the boundary of $T$. \\ Compute $\displaystyle\iint\limits_{S} \rho\textbf{F}\cdot\textbf{n}\ dA$ by both using the divergence theorem and directly by considering the flux on each face.\\

\item (35 pts) Let $f(x) = -x$ for $0<x<1$.  Find the Fourier series for an \textbf{odd} periodic extension of $f$, listing the first 4 non-zero terms.  Then, find the general solution to the ODE $y''+y=f(x)$.\\

\item (30 pts)  Use the Fourier sine transform to derive the solution formula for the heat equation $u_{t}=c^2u_{xx}$ on the half-infinite bar ($0\le x <\infty$) with Dirichlet boundary condition $u(0,t)=a$, for some constant $a$, and initial condition $u(x,0)=f(x)$.\\

\item (40 pts ) Once the temperature in an object reaches a steady state, the heat equation becomes the Laplace equation.  Use separation of variables to derive the steady-state solution to the heat equation on the rectangle $R = [0,1]\times[0,1]$ with the following Dirichlet boundary conditions: $u=0$ on the left and right sides; $u=f(x)$ on the bottom; $u=g(x)$ on the top.  That is, solve $u_{xx}+u_{yy}=0$ subject to $u(0,y) = u(1,y)=0$, $u(x,0) =f(x)$, and $u(x,1)=g(x)$.  You may assume the separation constant is negative: $F''/F = -k$, for $k>0$.  Finally, plot $u(x,y)$ when $f(x) = \sin{(\pi x)}$ and $g(x) = \sin{(3\pi x)}$.\\

\item (30 pts) Let $u(x,t)$ be the solution to $u_{tt}=16u_{xx}$ for $0\leq x\leq 2$ and $t\geq 0$, where: $u(0,t)=0$, $u(2,t)=0$, and $u(x,0) = f(x)=1-|x-1|$ for $0\le x\le 2$. Use D'Alembert's solution to find $u(1,0.1)$ and $u(1,0.6)$.  Be careful to consider that D'Alembert's solution uses the odd periodic extension of $f(x)$. \\

\item (30 pts)  Find the general solution of $u_{xx}-3u_{xy}+2u_{yy}=0$ using the  the method of characteristics: let $v=y+2x$ and $w=y+x$; define $U(v,w)$ to be $U(v,w) = U(y+2x,y+x)=u(x,y)$; derive and solve a PDE for $U(v,w)$; convert back to $u(x,y)$. Use your solution to provide a non-trivial example of a solution.  


\end{enumerate}


\end{document}



