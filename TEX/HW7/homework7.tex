\documentclass[10pt]{article}
\usepackage{amsmath,amsfonts,amssymb,amsthm}
\usepackage{graphicx,mathrsfs}
\pagestyle{empty}
%\textheight 9in
%\textwidth 6.5in
%\oddsidemargin 0pt
%\evensidemargin 0pt
%\headheight 0pt
%\topmargin -40pt
\usepackage[left=2cm, top=2cm, bottom = 2cm, right=2cm, nohead, nofoot]{geometry}


\begin{document}

\begin{center}
{\Large\bf  Homework - MTH 357}\\
\vspace{.1in}


\end{center}

\noindent Instructions:  %Below is a list of assigned exercises and supplemental exercises. Only the assigned exercises are to be turned in.  
Homework is to be neat and organized.  \textbf{If it's messy it's wrong.} Answers without the necessary supporting work are worth 0.   You may discuss problems with others but must always produce your own work and write your own solutions.  Copying someone else's homework is considered cheating.\\



\noindent \textbf{HW7, due 11/16}
\begin{enumerate}
\item Find the Fourier series representation of $u(x,t)$, the solution to the wave equation with $c=1$, $L=1$, $g(x)=0$, and $f(x)=0.01x(1-x)$ on $0\le x< 1$, extended as an odd function. List the first 4 non-zero terms. You may use Mathematica to find $B_n$ but only if you submit the notebook file showing your work.
\item Using your solution to \#1, plot $f(x)$, $u(x,0)$, and $u(x,0.5)$ on the same graph.
\item Using D'Alembert's solution to \#1 (i.e. $u(x,t)=\frac{1}{2}(f(x+t)+f(x-t))$), plot $u(x,t)$, $\frac{1}{2}f(x+t)$, and $\frac{1}{2}f(x-t)$ on the same graph, creating one graph  for each of $t = 0, 0.2., 0.4, 0.6, 0.8,$ and $1.$  You probably want to use Mathematica for this. Note, you MUST extend $f(x)$ to be odd with period 2.
\end{enumerate}



\end{document}
