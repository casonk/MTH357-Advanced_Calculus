\documentclass[10pt]{article}
\usepackage{amsmath,amsfonts,amssymb,amsthm}
\usepackage{graphicx,mathrsfs}
\pagestyle{empty}
%\textheight 9in
%\textwidth 6.5in
%\oddsidemargin 0pt
%\evensidemargin 0pt
%\headheight 0pt
%\topmargin -40pt
\usepackage[left=2cm, top=2cm, bottom = 2cm, right=2cm, nohead, nofoot]{geometry}


\begin{document}

\begin{center}
{\Large\bf  Homework - MTH 357}\\
\vspace{.1in}


\end{center}

\noindent Instructions:  
Homework is to be neat and organized.  \textbf{If it's messy it's wrong.} Answers without the necessary supporting work are worth 0.   You may discuss problems with others but must always produce your own work and write your own solutions.  Copying someone else's homework is considered cheating.\\

\noindent \textbf{HW4, due 10/28}\\
At the end of class on Thursday, we derived Fourier cosine and sine integrals for $e^{-kx}$ that matches for $x>0$: $e^{-kx}=\dfrac{2k}{\pi}\displaystyle\int_0^\infty \dfrac{\cos{wx}}{k^2+w^2}\ dw$ and $e^{-kx}=\dfrac{2}{\pi}\displaystyle\int_0^\infty \dfrac{w\sin{wx}}{k^2+w^2}\ dw$. These will be very helpful for the following two problems.
\begin{enumerate}
\item Represent $f(x)=\dfrac{1}{1+x^2}$ as a Fourier cosine integral.
\item Represent $g(x)=\dfrac{x}{1+x^2}$ as a Fourier sine integral.
\item Use Mathematica to plot the integral representations of $f$ and $g$ on the interval $[-3,3]$. Look at the end of the 11.7.nb file on Canvas for the easiest syntax to use, i.e., the ones where I define f using NIntegrate, and then Plot f.
\end{enumerate}

\vspace{.5in}
\noindent \underline{\textbf{Supplemental Exercises}}\\


\noindent 11.1: 12-22\\
11.2: 8-17, 23-28\\
11.3: 6-16\\
11.4: 2-8\\
11.7: 1-12, 16-20\\
11.8: 1-6, 9-13\\
11.9: 2-15\\

\noindent 12.1: 2-14\\
12.3: 5-14\\
12.4: 5-18\\
12.6: 5-10, 12-15, 21-22\\
12.7: 2-8\\
12.9: 4-8, 11-17\\







\end{document}
