\documentclass[10pt]{article}
\usepackage{amsmath,amsfonts,amssymb,amsthm}
\usepackage{graphicx,mathrsfs}
\pagestyle{empty}
%\textheight 9in
%\textwidth 6.5in
%\oddsidemargin 0pt
%\evensidemargin 0pt
%\headheight 0pt
%\topmargin -40pt
\usepackage[left=2cm, top=2cm, bottom = 2cm, right=2cm, nohead, nofoot]{geometry}


\begin{document}

\begin{center}
{\Large\bf  Homework - MTH 357}\\
\vspace{.1in}


\end{center}

\noindent Instructions:  
Homework is to be neat and organized.  \textbf{If it's messy it's wrong.} Answers without the necessary supporting work are worth 0.   You may discuss problems with others but must always produce your own work and write your own solutions.  Copying someone else's homework is considered cheating.\\


\noindent \textbf{HW8, due 12/7}
\begin{enumerate}
\item Find the general solution of $u_{xx}-6u_{xy}+9u_{yy}=0$ by the following: let $x=v$ and $y=w-3v$, or equivalently, $v=x$ and $w=y+3x$; define $U(v,w)$ to be $U(v,w) = u(v,w-3v)=u(x,y)$; derive and solve a PDE for $U(v,w)$; convert back to $u(x,y)$.  Hint: the solution will involve two arbitrary functions. Use you solution to provide a non-trivial example of a solution.

\item  Find the Fourier series representation of $u(x,t)$, the solution to the heat equation on a metal bar of length $L=10$ with $\rho=10.6$, $K=1.04$, $\sigma=0.056$,  and $u(x,0)=f(x)=4-0.8|x-5|$ on $0\le x\le 10$, extended as an odd function.  Enforce the Dirichlet boundary condition of $u(0,t)=u(10,t)=0$.
\item Use Mathematica and your solution from the last problem to plot $u(x,0)$ and $u(x,2)$ on the same graph.  When defining $u$, use the first four non-zero terms of the series.
\item Find the Fourier series representation of the steady-state solution $u(x,y)$ of the heat equation on the square metal plate with corners $(0,0)$ and $(2,2)$ in the plane, satisfying the following boundary conditions: $u_y(x,0) = u_x(0,y) = u_x(2,y)=0$ and $u(x,2)=\pi$.
\end{enumerate}







\end{document}
